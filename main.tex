\documentclass{article}

\title{Computer Workshop \\ Final home work}
\author{Hamidreza Yadegari}
\date{Summer of 4031}

\begin{document}
   \maketitle 

    \newpage
    
    \section{Git and GitHub}
        \subsection{Repository Initialization and Commits}
            First, we create a repository in github in such a way 
            that we enter the desired name in its creation and tick
             the box to create a README file. Then, with the 'git clone
              ssh' command, we transfer the above repository to our 
              system. Then create a file called main.tex to write the 
              answers to the questions and use the command 'git add.'  
              Take it to stage mode and commit the file with the command
               'git commit -m "add main.tex" and then apply the changes to 
               github with the command 'git push origin master'. Now the 
               project is ready for the next stage...

        
        \subsection{GitHub Actions for LaTeX Compilation}
        Following the answer to the previous question, in order to set Github
         action, we must first create a directory named .github in the main 
         directory of the remote repository and create a directory named workflows
          in .github. Then we create a file called latex-build.yml in the workflows
           and put the settings on the site \url{https://mrturkmen.com/posts/build-release-latex/}
            into it. We just have to be careful that in the root_file, asset_path, and asset_name
             fields, instead of the word main, we replace the name of the LaTeX file 
             created in the previous step. Now add the changes with the git add command.
              and finalize git commit -m "add .github" in the remote repository and apply
               the changes in github with the command 'git push origin main'.

\end{document}